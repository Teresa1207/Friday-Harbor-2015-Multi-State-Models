\documentclass[11pt]{article}
\usepackage{amsmath}
\title{Friday Harbour 2015, Rush data analysis 3}
\author{Terry Therneau}

\addtolength{\textwidth}{.5in}
\addtolength{\oddsidemargin}{-.255in}
\setlength{\evensidemargin}{\oddsidemargin}
\newcommand{\code}[1]{\texttt{#1}}

\usepackage{Sweave}
\begin{document}
\input{analysis4b-concordance}
%\SweaveOpts{concordance=TRUE}



\maketitle
\section{Notes}
  This is a reprise and extention of the original analysis, using some
of the tools in the newer release (2.39-4) of the R survival library.
In the .Rnw file from which the pdf is generated, change the knitr
option \code{echo} to TRUE in order to see all of the R code used to
create the report.  

\section{Data}
The data set is deficient in that we have the death age for those who
have died, but only the last cognitive visit for those still alive.
Verbally, I was reassured at the meeting that no one is lost to follow
up, at least with respect to death, and thus one could use the date on which
the data set was created as the last known alive date.
(This fact does us no practical good, however, since the enrollment date is
not included in the distributed data.)

Now if this follow-up statement were true and subjects faithfully come back
for yearly visits, 
then one would expect that the interval between
the last visit and death, for those who have died, would be approximately
uniformly distributed over a 0-12 month interval.

One nuisance: age in years is a long decimal value, truncated to a fixed number
of digits in the csv file.  This leads to some computed ages that should be the
same but differ by machine roundoff, which in turn caused some head scratching
at one point in the analysis.  Hence we first converted to an
integer age in days and used that for all intermediate calculations.

\section{Time scales}
In this study we have a choice of time-from-enrollment or time-from-birth (age)
as the underlying time scale for the Cox model.  
Which one should we choose?
A general rule of thumb for Cox models is to use the time scale that is
most important as the baseline time and model any others.
That is, if you had a choice to be told either how old a subject was or how
long they had been on study, which one would tell you more about their
expected event rate, i.e., death?   In this study I would argue for age
as the dominant scale.
By using age as the underlying time scale for our models, the compuatations 
at each event compare the subject who dies to all others of the 
same age as a matched set.  Age ceases to be a confounder.

In this data set we saw that time-since-enrollment is also a predictor due to
enrollment bias.
It can be modeled reasonably well as a linear term over the first 8 years
of follow-up.  
Thinking through how this should be included in the data sets and entered into
the modeling took up a fair bit of my thinking time, but in the end it seems
that I need not have worried so much. Follow-up year is mostly uncorrelated
with the covariates we care about, so adding ftime to the model makes 
only minor
changes in estimated effects.  
It does increase the overall goodness of fit for the model,
but that turns out to be secondary.

If instead we had used time-since-enrollment as the time scale, then 
the honeymoon effect is swept into the baseline hazard and can be ignored.
However, age then needs to be included in the models as a covariate.
Age is a dominating effect (for death), so one has to
model it correctly and with precision.  If other covariates of interest are
correlated with age, then any imprecision in the age model will badly confound 
estimates for these other variables.  Interactions of age and covariates
are also important terms.

One thing that age scale models cannot do is answer questions such as
``how much of the increase in deaths with age is due to increasing blood
pressure''?  This question views age as a bystander rather than a root cause.
A Cox model on age scale addresses instead the excess risk question: for
two subjects of the same age, how much does high blood pressure increase
the relative risk of death for one versus the other.  Age is firmly fixed as
the root and other covariates as modifiers.

\section{Fits}
